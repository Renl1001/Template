% !TEX program = xelatex
%==============================常用宏包、环境==============================%
\documentclass[twocolumn,a4]{article}%两栏,A4大小
%\documentclass[a4paper,11pt]{article}
\usepackage{xeCJK} % 中文支持
\usepackage{amsmath, amsthm}
\usepackage{listings,xcolor} %插入代码
\usepackage{geometry} % 设置页边距
\usepackage{fontspec}
\usepackage{graphicx}
\usepackage{fancyhdr} % 自定义页眉页脚
\usepackage[colorlinks]{hyperref}  % 目录可跳转
\setsansfont{Consolas} % 设置英文字体
\setmonofont[Mapping={}]{Consolas} % 英文引号之类的正常显示,相当于设置英文字体
\geometry{left=1cm,right=1cm,top=2cm,bottom=0.5cm} % 页边距
\setlength{\columnsep}{30pt} %两栏之间的间距大小
%\setlength\columnseprule{0.4pt} % 分割线
%==============================常用宏包、环境==============================%

%==============================页眉、页脚、代码格式设置==============================%
% 页眉、页脚设置
\pagestyle{fancy}
% \lhead{CUMTB}
\fancyhf{}
\fancyhead[C]{\CJKfamily{hei} SICNU ACM Template}
\lhead{\CJKfamily{hei} SICNU ACM Template}
\chead{}
% \rhead{Page \thepage}
\rhead{\CJKfamily{hei} 第 \thepage 页}
\lfoot{} 
\cfoot{}
\rfoot{}
%\renewcommand{\headrulewidth}{0.4pt} 
%\renewcommand{\footrulewidth}{0.4pt}

% 代码格式设置
\lstset{
    language    = c++,
    numbers     = left,
    numberstyle = \tiny,
    breaklines  = true,
    captionpos  = b,
    tabsize     = 4,
    frame       = shadowbox,
    columns     = fullflexible,
    commentstyle = \color[RGB]{0,128,0},
    keywordstyle = \color[RGB]{0,0,255},
    basicstyle   = \small\ttfamily,
    stringstyle  = \color[RGB]{148,0,209}\ttfamily,
    rulesepcolor = \color{red!20!green!20!blue!20},
    showstringspaces = false,
}
%==============================页眉、页脚、代码格式设置==============================%

%==============================标题和目录==============================%
\title{\CJKfamily{hei} \bfseries SICNU ACM Template}
\author{renli}
\renewcommand{\today}{\number\year 年 \number\month 月 \number\day 日}

\begin{document}\small
\begin{titlepage}
\maketitle
\end{titlepage}

\newpage
\pagestyle{empty}
\renewcommand{\contentsname}{目录}
\tableofcontents %生成目录
\newpage\clearpage
\newpage
\pagestyle{fancy}
\setcounter{page}{1}   %new page
%==============================标题和目录==============================%

%==============================正文部分==============================%
%0.头文件
\section{头文件}
\lstinputlisting{0_Include/0_Include.cpp}
%1.数学
\section{数学}
\subsection{素数}
    \subsubsection{素数筛}
    \lstinputlisting{1_Math/1_Prime/1_PrimeSieve.cpp}
    \subsubsection{区间筛}
    \lstinputlisting{1_Math/1_Prime/2_SegmentSieve.cpp}
    \subsubsection{Miller Rabin 素数判断}
    \lstinputlisting{1_Math/1_Prime/3_MillerRabin.cpp}
\subsection{高斯消元}
    $A^{-1} * |A| = (A^*)$
    \subsubsection{求逆矩阵}
    \lstinputlisting{1_Math/2_Gauss/1_InverseMatrix.cpp}
    \subsubsection{解01方程组}
    \lstinputlisting{1_Math/2_Gauss/2_01.cpp}
\subsection{矩阵快速幂}
    \indent
$F(i) = \begin{Bmatrix}F(i-1)+F(i-2)+i^3+i^2+i+1&i > 1\\0&i=0\\1&i=1\end{Bmatrix}$  求F(i)
\indent
$\begin{pmatrix}F(i)\\F(i-1)\\i^3\\i^2\\i\\1\end{pmatrix} = \begin{pmatrix}0&1&1&1&4&6&4\\1&0&0&0&0&0&0\\0&0&0&1&3&3&1\\0&0&0&0&1&2&1\\0&0&0&0&0&1&1\\0&0&0&0&0&0&1\end{pmatrix}*\begin{pmatrix}F(i-1)\\F(i-2)\\(i-1)^3\\(i-1)^2\\(i-1)\\1\end{pmatrix}$
    \lstinputlisting{1_Math/3_Matrix/1_MatrixFastPow.cpp}
%2.字符串
\section{字符串}
\subsection{KMP}
\lstinputlisting{2_String/1_kmp.cpp}
\subsection{扩展KMP}
\lstinputlisting{2_String/2_ekmp.cpp}
\subsection{Manacher 最长回文子串}
\lstinputlisting{2_String/3_Manacher.cpp}
\subsection{AC自动机}
\lstinputlisting{2_String/4_ACAutomaton.cpp}
\subsection{后缀数组}
\lstinputlisting{2_String/5_SuffixArray.cpp}
\subsection{后缀自动机}
\lstinputlisting{2_String/6_SuffixAutomation.cpp}
\subsection{字符串哈希}
\lstinputlisting{2_String/7_hash.cpp}
%3.数据结构
\section{数据结构}
\subsection{RMQ}
    \subsubsection{一维RMQ}
    \lstinputlisting{3_DataStructure/1_RMQ/1_RMQ1.cpp}
    \subsubsection{二维RMQ}
    \lstinputlisting{3_DataStructure/1_RMQ/2_RMQ2.cpp}
\subsection{线段树}
    \subsubsection{宏定义}
    \lstinputlisting{3_DataStructure/2_SegmentTree/0_Define.cpp}
    \subsubsection{单点修改}
    \lstinputlisting{3_DataStructure/2_SegmentTree/1_SinglePointUpdate.cpp}
    \subsubsection{区间修改}
    \lstinputlisting{3_DataStructure/2_SegmentTree/2_IntervalUpdate.cpp}
\subsection{分块}
\lstinputlisting{3_DataStructure/3_Block.cpp}
\subsection{离散化}
\lstinputlisting{3_DataStructure/4_discretization.cpp}
%4.图论
\section{图论}
\subsection{最小生成树}
    \subsubsection{并查集}
    \lstinputlisting{4_Graph/1_MinimalSpanningTree/0_UnionFindSet.cpp}
    \subsubsection{Kruskal}
    \lstinputlisting{4_Graph/1_MinimalSpanningTree/1_Kruskal.cpp}
    \subsubsection{Prim}
    \lstinputlisting{4_Graph/1_MinimalSpanningTree/2_Prim.cpp}
\subsection{最短路}
    \subsubsection{Dijkstra}
    \lstinputlisting{4_Graph/2_ShortestPath/1_Dijkstra.cpp}
    \subsubsection{SPFA}
    \lstinputlisting{4_Graph/2_ShortestPath/2_spfa.cpp}
    \subsubsection{Floyd}
    \lstinputlisting{4_Graph/2_ShortestPath/3_Floyd.cpp}
\subsection{LCA}
    \subsubsection{离线Tarjan}
    \lstinputlisting{4_Graph/3_LCA/1_Tarjan.cpp}
    \subsubsection{LCA 倍增法}
    \lstinputlisting{4_Graph/3_LCA/2_Doubly.cpp}
\subsection{拓扑排序}
\lstinputlisting{4_Graph/4_TopSort/1_TopSort.cpp}
\subsection{网络流}
    \subsubsection{建模技巧}
    \subsubsection*{建模技巧}


\indent

\textbf{二分图带权最大独立集}。给出一个二分图,每个结点上有一个正权值。要求选出一些点,使得这些点之间没有边相连,且权值和最大。

\indent

\textbf{解:}在二分图的基础上添加源点$S$和汇点$T$,然后从$S$向所有$X$集合中的点连一条边,所有$Y$集合中的点向$T$连一条边,容量均为该点的权值。$X$结点与$Y$结点之间的边的容量均为无穷大。这样,对于图中的任意一个割,将割中的边对应的结点删掉就是一个符合要求的解,权和为所有权减去割的容量。因此,只需要求出最小割,就能求出最大权和。

\indent

\textbf{公平分配问题}。把$m$个任务分配给$n$个处理器。其中每个任务有两个候选处理器,可以任选一个分配。要求所有处理器中,任务数最多的那个处理器所分配的任务数尽量少。不同任务的候选处理器集$\lbrace p_1 , p_2 \rbrace$保证不同。

\indent

\textbf{解:}本题有一个比较明显的二分图模型,即$X$结点是任务,$Y$结点是处理器。二分答案$x$,然后构图,首先从源点$S$出发向所有的任务结点引一条边,容量等于$1$,然后从每个任务结点出发引两条边,分别到达它所能分配到的两个处理器结点,容量为$1$,最后从每个处理器结点出发引一条边到汇点$T$,容量为$x$,表示选择该处理器的任务不能超过$x$。这样网络中的每个单位流量都是从$S$流到一个任务结点,再到处理器结点,最后到汇点$T$。只有当网络中的总流量等于$m$时才意味着所有任务都选择了一个处理器。这样,我们通过$O(\log m)$次最大流便算出了答案。

\indent

\textbf{区间$k$覆盖问题}。数轴上有一些带权值的左闭右开区间。选出权和尽量大的一些区间,使得任意一个数最多被k个区间覆盖。

\indent

\textbf{解:}本题可以用最小费用流解决,构图方法是把每个数作为一个结点,然后对于权值为$w$的区间$[u,v)$加边$u→v$,容量为$1$,费用为$-w$。再对所有相邻的点加边$i→i+1$,容量为$k$,费用为$0$。最后,求最左点到最右点的最小费用最大流即可,其中每个流量对应一组互不相交的区间。如果数值范围太大,可以先进行离散化。

\indent

\textbf{最大闭合子图}。给定带权图$G$(权值可正可负),求一个权和最大的点集,使得起点在该点集中的任意弧,终点也在该点集中。

\indent

\textbf{解:}新增附加源$s$和附加汇$t$,从$s$向所有正权点引一条边,容量为权值;从所有负权点向汇点引一条边,容量为权值的相反数。求出最小割以后,$S - \lbrace s \rbrace$就是最大闭合子图。

    \subsubsection{Edge}
    \lstinputlisting{4_Graph/5_NetworkFlow/1_Edge.cpp}
    \subsubsection{Dinic}
    \lstinputlisting{4_Graph/5_NetworkFlow/2_Dinic.cpp}
    \subsubsection{ISAP}
    \lstinputlisting{4_Graph/5_NetworkFlow/3_ISAP.cpp}
    \subsubsection{MCMF}
    \lstinputlisting{4_Graph/5_NetworkFlow/4_MCMF.cpp}
%5.计算几何
\section{计算几何}
\subsection{基本函数}
    \subsubsection{定义点和线}
    \lstinputlisting{5_Geometry/0_Basic/0_Basic.cpp}
    \subsubsection{两点间距离}
    \lstinputlisting{5_Geometry/0_Basic/1_Point-Point.cpp}
    \subsubsection{线段相交}
    \lstinputlisting{5_Geometry/0_Basic/2_Segment-Segment.cpp}
    \subsubsection{直线和线段相交}
    \lstinputlisting{5_Geometry/0_Basic/3_Line-Segment.cpp}
    \subsubsection{点到直线距离}
    \lstinputlisting{5_Geometry/0_Basic/4_Point-Line.cpp}
    \subsubsection{点到线段距离}
    \lstinputlisting{5_Geometry/0_Basic/5_Point-Segment.cpp}
    \subsubsection{判断点在线段上}
    \lstinputlisting{5_Geometry/0_Basic/6_PointOnSegment.cpp}
\subsection{多边形}
    \subsubsection{计算多边形面积}
    \lstinputlisting{5_Geometry/1_Polygon/1_Area.cpp}
    \subsubsection{判断点在凸多边形内}
    \lstinputlisting{5_Geometry/1_Polygon/2_PointInConvex.cpp}
    \subsubsection{判断点在任意多边形内}
    \lstinputlisting{5_Geometry/1_Polygon/3_PointInPolygon.cpp}
    \subsubsection{判断凸多边形}
    \lstinputlisting{5_Geometry/1_Polygon/4_JudgeConvex.cpp}
    \subsubsection{凸包}
    \lstinputlisting{5_Geometry/1_Polygon/5_Graham.cpp}
\subsection{圆}
    \subsubsection{外心}
    \lstinputlisting{5_Geometry/2_Circle/1_waixin.cpp}
    \subsubsection{两圆相交的面积}
    \lstinputlisting{5_Geometry/2_Circle/2_Area_of_overlap.cpp}
%6.动态规划
\section{动态规划}
\subsection{最长上升子序列}
\lstinputlisting{6_DynamicProgramming/1_LongestIncrease.cpp}
\subsection{数位dp}
\lstinputlisting{6_DynamicProgramming/2_DigitStatistics.cpp}
%7.其他
\section{其他}
\subsection{Java}
\lstinputlisting{7_Others/1_Java/1_Java.java}
\subsection{STL}
    \subsubsection{优先队列}
    \lstinputlisting{7_Others/2_STL/1_priority_queue.cpp}
\subsection{SG函数}
    \subsubsection{解题模型}


\indent

1. 把原游戏分解成多个独立的子游戏,则原游戏的SG函数值是它的所有子游戏的SG函数值的异或。\\
即 \textbf{$sg(G)=sg(G1) \wedge sg(G2) \wedge ... \wedge sg(Gn)$}。

\indent

2. 分别考虑没一个子游戏,计算其SG值。
    SG值的计算方法:(重点)\\
    1.可选步数为$1-m$的连续整数,直接取模即可,$SG(x) = x \% (m+1)$;\\
    2.可选步数为任意步,$SG(x) = x$;\\
    3.可选步数为一系列不连续的数,用模板计算。\\

\indent
一般DFS只在打表解决不了的情况下用,首选打表预处理。

    \subsubsection{打表}
    \lstinputlisting{7_Others/3_SG/1_table.cpp}
    \subsubsection{dfs}
    \lstinputlisting{7_Others/3_SG/2_dfs.cpp}
\subsection{战术研究}
    - 读新题的优先级高于一切\\
    - 读完题之后必须看一遍clarification\\
    - 交题之前必须看一遍clarification\\
    - 可能有SPJ的题目提交前也应该尽量做到与样例输出完全一致\\
    - A时需要检查INF是否设小\\
    - 构造题不可开场做\\
    - 每道题需至少有两个人确认题意\\
    - 上机之前做法需得到队友确认\\
    - 带有猜想性质的算法应放后面写\\
    - 当发现题目不会做但是过了一片时应冲一发暴力\\
    - 将待写的题按所需时间放入小根堆中,每次选堆顶的题目写\\
    - 交完题目后立马打印随后让出机器\\
    - 写题超过半小时应考虑是否弃题\\
    - 细节、公式等在上机前应在草稿纸上准备好,防止上机后越写越乱\\
    - 提交题目之前应检查$solve(n,m)$是否等于$solve(m,n)$\\
    - 检查是否所有东西都已经清空\\
    - 对于中后期题应该考虑一人写题,另一人在一旁辅助,及时发现手误\\
    - 最后半小时不能慌张\\
    - 对于取模的题,在输出之前一定要再取模一次进行保险\\
\subsection{打表找规律方法}
    - 直接找规律\\
    - 差分后找规律\\
    - 找积性\\
    - 点阵打表\\
    - 相除\\
    - 找循环节\\
    - 凑量纲\\
    - 猜想满足$P(n)f(n)=Q(n)f(n−2)+R(n)f(n−1)+C$,其中$P,Q,R$都是关于$n$的二次多项式\\
%==============================正文部分==============================%
\end{document}